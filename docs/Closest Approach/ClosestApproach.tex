\documentclass[times,12pt]{article}
\usepackage{parskip}
\usepackage[margin=1in]{geometry}
\usepackage{mathtools}
\usepackage{xcolor}

\DeclareMathOperator{\wptm}{time}
\DeclareMathOperator{\sort}{sort}

\newcommand{\figh}[3][0.4]{
	\begin{figure}[tbp]
		\begin{center}
			\includegraphics[width=#1\textwidth]{diagrams/#2.png}%
		\end{center}
		\caption{#3}
		\label{fig:#2}
	\end{figure}
}

\title{Technical Note: Determining the Closest Approaches between Two Piece-wise Linear Routes}
\author{Stephen Fitzpatrick\\{\normalsize Kestrel Institute}}

\date{August 2019}

\begin{document}
\maketitle

\begin{abstract}
	This report details how to determine the points of closest approach of two objects, each moving along a piece-wise linear route.
\end{abstract}

\section{Introduction}
The motion of an object may be specified in terms of a \textit{route}, that is, a sequence of \textit{way points}, in which each way point provides a \emph{position in space} (in arbitrary dimensions) and a \emph{time} (with time strictly increasing between successive way points). The object has a constant velocity between successive way points, and thus moves along a sequence of straight line segments between successive way points. For example, Table~\ref{fig:a-route} shows a 2-dimensional route with way points at times 0, 10, 20 \& 60.

\begin{table}[htbp]
	\centering
	\begin{tabular}[t]{|r|c|} \hline
		Time & Position \\ \hline
		0 & (0, 0) \\ \hline
		10 & (2, 2) \\ \hline
		20 & (4, 2) \\ \hline
		60 & (0, 0) \\ \hline
	\end{tabular}
\caption{A route as a sequence of way points}
\label{fig:a-route}
\end{table}

Given two objects, each moving along its own route, the distance between the objects typically varies over time. (The distance between two positions is assumed to be \textit{Euclidean}~--- that is, the square root of the sum of the squares of the differences along each dimension of the objects' positions.)

This report details how to determine the positions and times at which the distance between the objects is minimized as they progress along their routes~--- that is, the routes' \textit{closest approaches}. Note that this is the \emph{instantaneous} distance that is minimized; an alternative is the smallest distance between the all of the positions covered by one route, without regard to time, and all of the positions covered by the other route, without regard to time. The latter is not considered in this note.

First, the closest approach is determined for two objects, each moving along a single straight line segment, with the two motions having common start times and common end times. This is then generalized to piece-wise linear routes.

\section{Closest Approach between Two Objects Moving Along Straight Line Segments with Common Start Times and Common End Times}
Given an object moving with constant velocity from starting position $\vec{s}$ at time $t_s$ to ending position $\vec{e}$ at time $t_e > t_s$, the object's position at some intermediate time $t$ ($t_s \le t \le t_e$) is given by:
\begin{equation}
\label{eqn:interpolate-position}
	p(k) = \vec{s} + k(\vec{e} - \vec{s})
\end{equation}
where $k = (t - t_s)/(t_e - t_s)$, the \textit{interpolation parameter}, is the fraction of the route covered up to time $t$.

Note that $k=0$ at $t=t_s$, $0 < k < 1$ for $t_s < t < t_e$, and $k=1$ at $t=t_e$. That is, $k$ has value 0 for the start of the motion, 1 for the end of the motion, and intermediate values in between. Since $k$ is linear in $t$, the object's motion is linear in $k$.

\subsection{Distance between Objects}

For two objects, with starting positions $\vec{s_i}$ and ending positions $\vec{e_i}$, for $i=1$ and $i=2$, and with common starting time $t_s$ and common ending time $t_e$, the difference in the objects' positions at a given $k$ ($0 \le k \le 1$) is:
\[
\begin{array}{lll}
	\Delta{\vec{p}}(k) & = & \vec{p_1}(k) - \vec{p_2}(k) \\
	& = & \vec{s_1} + k(\vec{e_1} - \vec{s_1}) - \vec{s_2} - k(\vec{e_2} - \vec{s_2}) \\
	& = & (\vec{s_1} - \vec{s_2}) + k[(\vec{e_1} - \vec{e_2}) - (\vec{s_1} - \vec{s_2})] \\
	& = & \Delta{\vec{s}} + k(\Delta{\vec{e}} - \Delta{\vec{s}})
\end{array}
\]
where $\Delta{\vec{s}} = \vec{s_1} - \vec{s_2}$ is the (vector) difference between the starting points, and $\Delta{\vec{e}} = \vec{e_1} - \vec{e_2}$ is the (vector) difference between the ending points.

Note that $\Delta{\vec{s}} + k(\Delta{\vec{e}} - \Delta{\vec{s}})$ is the
interpolation, by $k$, between the objects' initial separation,
$\Delta{\vec{s}}$, and their final separation, $\Delta{\vec{e}}$.

If the two objects are moving in parallel, then the difference in their positions is constant. In particular, $\Delta{\vec{e}} - \Delta{\vec{s}} = \vec{0}$, so $\Delta{\vec{p}}(k)$ is constant, with value $\Delta{\vec{s}}$.

The \emph{distance} between the objects, $D(k)$, is
\begin{equation}
\label{eqn:distance}
	D(k) = ||\Delta{\vec{p}}(k)||
\end{equation}
where $||v|| = \sqrt{\sum_{j=1}^n (v_j)^2}$ is the 2-norm of the vector $\vec{v}$, where $\vec{v}$ has $n$ dimensions, and $v_j$ denotes the $j^{th}$ dimension of $\vec{v}$ (for $j=1, \dots, n$).

For example, Figure~\ref{fig:PositionsForK} shows the movements of two objects:
\begin{itemize}
	\item One object follows the blue line, starting at the left end and moving right.
	\item The second object follows the red line, starting at the bottom and moving up.
	\item The green, dashed lines show the offsets between the two objects for various values of $k$.
\end{itemize}

\figh[0.4]{PositionsForK}{Closest approach between two objects moving in straight lines}

Figure~\ref{fig:Distances} shows how the distance between the objects varies as $k$ varies. Note $D(k)$ is the instantaneous distance between the objects ($k$ being linear in $t$)~--- this is the distance for which the minimum is to be found.

In contrast, the atemporal distance can be defined as the minimum distance between any position covered by one route and any position covered by the other route:
\[
\min_{p_1 \in P_1, p_2 \in P_2} ||p_1 - p_2||
\]
where $P_i = \{ p_i(k) \ | \ 0 \le k \le 1 \}$, for $i=1$ or $i=2$, is the set of all positions covered by route $i$.

In Figure~\ref{fig:PositionsForK}, the instantaneous distance has a minimum at $k \approx 0.65$, with a value of 0.30, whereas the atemporal distance between the two routes is 0 since they cross. The crossing point is \emph{not} the closest approach because the two objects are not at the crossing point at the same time.

\figh[0.6]{Distances}{Distance between objects as a function of $k$; minimum is at $k \approx 0.65$}

\subsection{Computing the Minimum Distance}
Minimizing the inter-object distance can be achieved by minimizing the square of the distance, which is simpler.

The square of the distance is:
\[
\begin{array}{lll}
	D^2(k) & = & ||\Delta{\vec{p}}(k)||^2 \\
	& = & \sum_{j=1}^n [\Delta{s}_j + k(\Delta{e}_j - \Delta{s}_j)]^2 \ .
\end{array}
\]

Differentiating with respect to $k$:
\[
	\frac{\partial D^2(k)}{\partial k} = \sum_{j=1}^n 2[\Delta{s}_j + k(\Delta{e}_j - \Delta{s}_j)](\Delta{e}_j - \Delta{s}_j) \ .
\]

At the minimum, the derivative is 0, so:
\[
	\sum_{j=1}^n k(\Delta{e}_j - \Delta{s}_j)(\Delta{e}_j - \Delta{s}_j)
	= - \sum_{j=1}^n \Delta{s}_j (\Delta{e}_j - \Delta{s}_j) \\
\]
which gives $k$ for the closest approach as:
\[
	k = \frac{\sum_{j=1}^n \Delta{s}_j (\Delta{s}_j - \Delta{e}_j)}
	{\sum_{j=1}^n (\Delta{s}_j - \Delta{e}_j)^2} \\
 	= \frac{\Delta{\vec{s}} . (\Delta{\vec{s}} - \Delta{\vec{e}})}
	{||\Delta{\vec{s}} - \Delta{\vec{e}} ||^2}
\]
where $\vec{u} . \vec{v} = \sum_{j=1}^n u_jv_j$ is the inner product of vectors $\vec{u}$ and $\vec{v}$.
         
Note that $||\Delta{\vec{s}} - \Delta{\vec{e}} ||^2$ is the squared length of $\Delta{\vec{s}} - \Delta{\vec{e}}$. If this is 0, then the difference in the starting positions is the same as the difference in the ending positions; i.e., the objects are moving in parallel. In this case, $k$ is undefined by the above equation~--- the distance is the same for all k ($0 \le k \le 1$). If only the distance of closest approach is required (and not the positions or time), then the distance can be computed from, say, the starting positions or the ending positions.

Also note that the above equation may give $k < 0$ or $k > 1$. In such cases, if the objects' straight-line motions were to be extended before $t_s$ or after $t_e$, respectively, then the closest approach would occur in the extension. For example, Figure~\ref{fig:KGreaterThan1} shows two motions for which, if they were extended, $k \approx 1.4$ would give the closest approach. 

\figh[0.2]{KGreaterThan1}{Closest approach is outside the time interval}

If the equation gives $k<0$ then the closest approach within the time period $[t_s, t_e]$ occurs at $t_s$; i.e., at $k=0$. Likewise, if $k>1$, the closest approach within $[t_s, t_e]$ occurs at $t_e$; i.e., at $k=1$.

Thus, the closest approach within $[t_s, t_e]$ is given by:
\begin{equation}
\label{eqn:k-for-closest-approach}
	k = \max\left(0, \min\left(1,
		\frac{\Delta{\vec{s}} . (\Delta{\vec{s}} - \Delta{\vec{e}})}
		{||\Delta{\vec{s}} - \Delta{\vec{e}} ||^2}\right)\right) \ .
\end{equation}

Once $k$ has been determined for the closest approach, the position of each object at closest approach can be determined using Equation~\ref{eqn:interpolate-position}, and thence the inter-object distance at closest approach using Equation~\ref{eqn:distance}.  

\section{Closest Approaches along Piece-wise Linear Routes}
Equation~\ref{eqn:k-for-closest-approach} applies to two motions, each of which is a single straight-line segment; the two motions must have a common start time and a common end time.

A route comprises a sequence of straight-line segments, so, given two routes, $R_1$ and $R_2$, Equation~\ref{eqn:k-for-closest-approach} potentially could be applied to the first segment in $R_1$ and the first segment in $R_2$, then to the second segment in $R_1$ and the second segment in $R_2$, and so on, to determine the closest approaches segment by segment.

However, in general, the $j^{th}$ segment in $R_1$ will not have the same start time, nor the same end time, as the $j^{th}$ segment in $R_2$. Indeed, in general, $R_1$ and $R_2$ will not have the same number of segments. For example, Figure~\ref{fig:unaligned-routes} shows two routes, one with way points at times 0, 10, 20 \& 60; the other with way points at times 0, 20 \& 40.

To apply Equation~\ref{eqn:k-for-closest-approach}, the routes need to be \textit{aligned} in time, so that: (1) they have the same number of way points; and (2) the $j^{th}$ way points in each have the same time.

For example, Figure~\ref{fig:aligned-routes} shows the same two routes after alignment:
\begin{itemize}
	\item For Route 1, the straight line segment from position (4, 2) at time 20 to position (0, 0) a time 60, was split into two segments by inserting a way point at time 40 at position (2, 1), determined by interpolating the segment to time 40.
	
	\item Likewise, for Route 2, the segment from (2, 8) at time 0 to (0, 6) at time 20 was split by adding a new way point at time 10, with position (1, 7).
\end{itemize}

In both cases, the position of the new way point was determined by interpolating the original route to the appropriate time~--- thus, the insertion of the way point does not change the motion of the object following the route. 

Note that a way point at time 60 is \emph{not} inserted into Route 2 since Route 2 finishes at time 40. The final way point in Route 1 is discarded for the purpose of determining closest approaches.

\begin{table}[tbp]
	\centering
	\begin{tabular}[t]{|c|c|} \hline
		Time & Route 1 Position \\ \hline
		0 & (0, 0) \\ \hline
		10 & (2, 2) \\ \hline
		20 & (4, 2) \\ \hline
		60 & (0, 0) \\ \hline
	\end{tabular}
\qquad
\begin{tabular}[t]{|c|c|} \hline
	Time & Route 2 Position \\ \hline
	0 & (2, 8) \\ \hline
	20 & (0, 6) \\ \hline
	40 & (4, 2) \\ \hline
\end{tabular}
	\caption{Two routes with unaligned times}
	\label{fig:unaligned-routes}

\vspace{3ex}

\begin{tabular}{|c|c|c|} \hline
	Time & Route 1 Position & Route 2 Position \\ \hline
	0 & (0, 0) & (2, 8) \\ \hline
	10 & (2, 2) & \textbf{\color{blue}{(1, 7)}} \\ \hline
	20 & (4, 2) & (0, 6) \\ \hline
	40 & \textbf{\color{blue}{(2, 1)}} & (4, 2) \\ \hline
	\color{gray}{60} & \color{gray}{(0, 0)} & \\ \hline
\end{tabular}
\caption{Two routes with aligned times}
\label{fig:aligned-routes}
\end{table}

The following section provides methods for aligning the routes and determining the closest approaches. Note that, in theory, there may be multiple closest approaches, with the same distance.

\subsection{Aligning Times and Determining Closest Approaches}
First note that a straight-line movement from a way point with starting position $\vec{s}$ at time $t_s$ and ending position $\vec{e}$ at time $t_e$ can be broken into two consecutive movements by introducing an intermediate way point at any time $\tau$ between the starting and end times, with corresponding position given by interpolation between the starting and ending positions.

That is, the movement $(t_s, \vec{s}) \rightarrow (t_e, \vec{e})$ can be broken into the movements $(t_s, \vec{s}) \rightarrow (\tau, \vec{p}(\tau))$ and $(\tau, \vec{p}(\tau)) \rightarrow (t_e, \vec{e})$, where $t_s < \tau < t_e$ and $p(\tau) = \vec{s} + k(\vec{e} - \vec{s})$ where $k = (\tau - t_s) / (t_e - t_s)$.

Now assume initially that the two routes start at the same time and end at the same time. Then a sequence of increasing times $\tau_l$, $1 \le l \le m$ for some $m$, needs to be determined such that $\tau_1$ is the common start time, $\tau_m$ is the common end time, and for each successive pair of times $\tau_l$ and $\tau_{l+1}$, $1 \le l < m$, each object has a uniform velocity over the period $[\tau_l, \tau_{l+1}]$. Then Equation~\ref{eqn:k-for-closest-approach} can be used to determine the closest approach that occurs within each $[\tau_l, \tau_{l+1}]$, and the overall closest approaches determined by finding those that have the smallest separation.

\begin{table}[htbp]
	\centering
	\begin{tabular}{|r|r||c|c||c|c||c|c|c|c|} \hline
		\multicolumn{2}{|c||}{} & \multicolumn{2}{|c||}{Route 1} & \multicolumn{2}{|c||}{Route 2} & \multicolumn{4}{c|}{Closest Approach} \\ \hline
		$t_s$ & $t_e$ & $s_1$ & $e_1$ & $s_2$ & $e_2$ & Time & Route 1 & Route 2 & Distance \\ \hline
		0 & 10 & (0, 0) & (2, 2) & (2, 8) & (1, 7) & 10 & (2.0, 2.0) & (1.0, 7.0) & 5.10\\
		10 & 20 & (2, 2) & (4, 2) & (1, 7) & (0, 6) & 12 & (2.4, 2.0) & (0.8, 6.8) & 5.06\\
		20 & 40 & (4, 2) & (2, 1) & (0, 6) & (4, 2) & 36 & (2.4, 1.2) & (3.2, 2.8) & 1.79 \\ \hline
	\end{tabular}
\caption{Closest approaches for corresponding segments of aligned routes}
\label{fig:closest-approaches}
\end{table}

For example, Table~\ref{fig:closest-approaches} shows corresponding segments of aligned routes, with the closest approach for each segment. The rightmost columns of the table show the time at which the closest approach occurs in the segment, the way point in Route 1 for closest approach, the way point in Route 2 for closest approach, and the separation at closest approach. The overall closest approach thus occurs in the final segment (between times 20 and 40).

If the objects' routes do not start at the same time, then any motion prior to the later start time is ignored. Likewise, any motion after the earlier end time is ignored. In other words, the objects' separation is only considered over the common period of their two routes. If there is no common period (i.e., one route ends before the other begins), then there is no closest approach.

The algorithm for aligning two routes in time is detailed in Figure~\ref{fig:algorithm-for-aligning-routes}. Determining the closest approaches between two routes is detailed in Figure~\ref{fig:algorithm-for-closest-approaches}.

\begin{figure}[htb]
	\begin{enumerate}
		\item Given routes $R_1$ and $R_2$.
		
		\item Determine the common time period $[s, e]$ for the two routes: $s = \max(s_1, s_2)$ where $s_i$ is the start time of Route $R_i$; $e = \min(e_1, e_2)$ where $e_i$ is the end time of Route $R_i$.
		
		\item If there is no common period (i.e., $e < s$), then there is no closest approach and the algorithm is complete. The aligned routes are both empty. (Note that if one route is empty, so is the other.)
		
		\item Otherwise, let $T_i$ be the set of times of $R_i$'s way points that fall within $[s, e]$.
			
		\item Let $T$ be the sequence of times that contains, in ascending order, the time $s$, every time in $T_1$, every time in $T_2$, and the time $e$, with each time occurring at most once (i.e., $T = \sort(\{s\} \cup T_1 \cup T_2 \cup \{e\})$).
				
		\item Let $R'_i$ be the route formed by interpolating $R_i$ at each time in $T$. That is, the $j^{th}$ way point in $R'_i$ has time $T_j$, and position given by interpolating $R_i$ to time $T_j$.
		
		\item The aligned routes are $R'_1$ and $R'_2$.
	\end{enumerate}
\caption{Algorithm for aligning two routes in time}
\label{fig:algorithm-for-aligning-routes}
\end{figure}

\begin{figure}[htb]
	\begin{enumerate}
		\item Given two routes $R_1$ and $R_2$.
		
		\item Determine time-aligned routes $R'_1$ and $R'_2$.
		
		\item If both aligned routes are empty, there is no closest approach. The algorithm is complete.
		
		\item Otherwise, if either aligned route has exactly one way point, then that way point determines the unique closest approach. (This can happen only when one route's end time is the same as the other route's start time.)
		
		Let $t$ be the time of that way point~--- note that the other route must also have a way point at time $t$, since the routes are aligned in time. The positions of these two way points, at time $t$, are the positions of the closest approach, and their separation gives the distance of closest approach. The algorithm is complete.
		
		\item Otherwise, both aligned routes have at least two way points. Let $T$ denote the sequence of common times of the aligned routes, in ascending order.
		
		\item Determine a sequence $C_l$ of candidate closest approaches by considering each consecutive pair of times in $T$, as follows:
		
		\begin{enumerate}
			\item Let $t_s$ be the earlier of the pair of times. Let $t_e$ be the later of the pair of times.
			
			\item Let $s_i$ be the position in Route $R'_i$ at time $t_s$.
			
			\item Let $e_i$ be the position in Route $R'_i$ at time $t_e$.
			
			\item Determine the closest approach for the straight-line movements given by $t_s$, $s_i$, $t_e$ and $e_i$, using Equation~\ref{eqn:k-for-closest-approach}. If the movements are parallel (or effectively so given finite precision computations), then take the closest approach as occurring at the midpoint; i.e., at $t = (t_s + t_e)/2$.
			
		\end{enumerate}
	
	\item Determine the candidate closest approaches that have the smallest inter-object separation. These are the closest approaches between the two routes.
	\end{enumerate}

\caption{Algorithm for determining closest approaches}
\label{fig:algorithm-for-closest-approaches}
\end{figure}

\section{Conclusion}
This technical note details an algorithm for determining the times, positions and distance of closest approaches between two piece-wise linear routes.
\end{document}